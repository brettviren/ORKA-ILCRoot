\section{String base object and event level tags}
\label{App:StringCuts}
\subsection{Variables for run cuts}
{\ttfamily \noindent
\begin{verbatim}
  Int_t   fIlcRunId;                  //the run id
  Float_t fIlcMagneticField;          //value of the magnetic field
  Int_t   fIlcRunStartTimeMin;        //minimum run start date
  Int_t   fIlcRunStartTimeMax;        //maximum run start date
  Int_t   fIlcRunStopTimeMin;         //minmum run stop date
  Int_t   fIlcRunStopTimeMax;         //maximum run stop date
  TString fIlcrootVersion;              //ilcroot version
  TString fRootVersion;                 //root version
  TString fGeant3Version;               //geant3 version
  Bool_t  fIlcRunQuality;             //validation script
  Float_t fIlcBeamEnergy;             //beam energy cm
  TString fIlcBeamType;               //run type (pp, AA, pA)
  Int_t   fIlcCalibrationVersion;     //calibration version  
  Int_t   fIlcDataType;               //0: simulation -- 1: data  
\end{verbatim}
}
\subsection{Variables for event cuts}
To invoke one of these cuts, please make sure to use the {\ttfamily fEventTag.} identifier. Example: {\ttfamily "fEventTag.fNParticipants < 100"}.

\par
{\ttfamily \noindent
\begin{verbatim}
  Int_t fNParticipantsMin, fNParticipantsMax;
  Float_t fImpactParamMin, fImpactParamMax;

  Float_t fVxMin, fVxMax;
  Float_t fVyMin, fVyMax;
  Float_t fVzMin, fVzMax;
  Int_t fPrimaryVertexFlag;
  Float_t fPrimaryVertexZErrorMin, fPrimaryVertexZErrorMax;

  ULong64_t fTriggerMask;
  UChar_t fTriggerCluster;
  
  Float_t fZDCNeutron1EnergyMin, fZDCNeutron1EnergyMax;
  Float_t fZDCProton1EnergyMin, fZDCProton1EnergyMax;
  Float_t fZDCNeutron2EnergyMin, fZDCNeutron2EnergyMax;
  Float_t fZDCProton2EnergyMin, fZDCProton2EnergyMax;
  Float_t fZDCEMEnergyMin, fZDCEMEnergyMax;
  Float_t fT0VertexZMin, fT0VertexZMax;

  Int_t fMultMin, fMultMax;
  Int_t fPosMultMin, fPosMultMax;
  Int_t fNegMultMin, fNegMultMax;
  Int_t fNeutrMultMin, fNeutrMultMax;
  Int_t fNV0sMin, fNV0sMax;
  Int_t fNCascadesMin, fNCascadesMax;
  Int_t fNKinksMin, fNKinksMax;
  
  Int_t fNPMDTracksMin, fNPMDTracksMax;
  Int_t fNFMDTracksMin, fNFMDTracksMax;
  Int_t fNPHOSClustersMin, fNPHOSClustersMax;
  Int_t fNEMCALClustersMin, fNEMCALClustersMax;
  Int_t fNJetCandidatesMin, fNJetCandidatesMax;

  Float_t fTopJetEnergyMin;
  Float_t fTopNeutralEnergyMin;
  
  Int_t fNHardPhotonCandidatesMin, fNHardPhotonCandidatesMax;
  Int_t fNChargedAbove1GeVMin, fNChargedAbove1GeVMax;
  Int_t fNChargedAbove3GeVMin, fNChargedAbove3GeVMax;
  Int_t fNChargedAbove10GeVMin, fNChargedAbove10GeVMax;
  Int_t fNMuonsAbove1GeVMin, fNMuonsAbove1GeVMax;
  Int_t fNMuonsAbove3GeVMin, fNMuonsAbove3GeVMax;
  Int_t fNMuonsAbove10GeVMin, fNMuonsAbove10GeVMax;
  Int_t fNElectronsAbove1GeVMin, fNElectronsAbove1GeVMax;
  Int_t fNElectronsAbove3GeVMin, fNElectronsAbove3GeVMax;
  Int_t fNElectronsAbove10GeVMin,fNElectronsAbove10GeVMax;
  Int_t fNElectronsMin, fNElectronsMax;
  Int_t fNMuonsMin, fNMuonsMax;
  Int_t fNPionsMin, fNPionsMax;
  Int_t fNKaonsMin, fNKaonsMax;
  Int_t fNProtonsMin, fNProtonsMax;
  Int_t fNLambdasMin, fNLambdasMax;
  Int_t fNPhotonsMin, fNPhotonsMax;
  Int_t fNPi0sMin, fNPi0sMax;
  Int_t fNNeutronsMin, fNNeutronsMax;
  Int_t fNKaon0sMin, fNKaon0sMax;
  Float_t fTotalPMin, fTotalPMax;
  Float_t fMeanPtMin, fMeanPtMax;
  Float_t fTopPtMin;
  Float_t fTotalNeutralPMin, fTotalNeutralPMax;
  Float_t fMeanNeutralPtMin, fMeanNeutralPtMax;
  Float_t fTopNeutralPtMin;
  Float_t fEventPlaneAngleMin, fEventPlaneAngleMax;
  Float_t fHBTRadiiMin, fHBTRadiiMax;
\end{verbatim}
}